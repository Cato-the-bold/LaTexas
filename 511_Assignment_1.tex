%%%%%%%%%%%%%%%%%%%%%%%%%%%%%%%%%%%%%%%%%
% Short Sectioned Assignment
% LaTeX Template
% Adapted from a template by
% Frits Wenneker (http://www.howtotex.com)
%
% License:
% CC BY-NC-SA 3.0 (http://creativecommons.org/licenses/by-nc-sa/3.0/)
%
%%%%%%%%%%%%%%%%%%%%%%%%%%%%%%%%%%%%%%%%%

%----------------------------------------------------------------------------------------
%	PACKAGES AND OTHER DOCUMENT CONFIGURATIONS
%----------------------------------------------------------------------------------------

\documentclass[paper=a4, fontsize=12pt]{scrartcl} % A4 paper and 11pt font size

\usepackage[T1]{fontenc} % Use 8-bit encoding that has 256 glyphs
\usepackage{fourier} % Use the Adobe Utopia font for the document - comment this line to return to the LaTeX default
\usepackage[utf8]{inputenc}
\usepackage[english]{babel} % English language/hyphenation
\usepackage{amsmath,amsfonts,amsthm} % Math packages

\usepackage[ruled,linesnumbered,noend]{algorithm2e} % For typesetting algorithms

\usepackage{graphicx} % For including figures

\usepackage{sectsty} % Allows customizing section commands
\allsectionsfont{\normalfont\scshape} % Make all sections centered, the default font and small caps

\newtheorem{theorem}{Theorem}[section]
\newtheorem{corollary}{Corollary}[theorem]
\newtheorem{lemma}[theorem]{Lemma}

\usepackage{fancyhdr} % Custom headers and footers
\pagestyle{fancyplain} % Makes all pages in the document conform to the custom headers and footers
\fancyhead{} % No page header - if you want one, create it in the same way as the footers below
\fancyfoot[L]{} % Empty left footer
\fancyfoot[C]{} % Empty center footer
\fancyfoot[R]{\thepage} % Page numbering for right footer
\renewcommand{\headrulewidth}{0pt} % Remove header underlines
\renewcommand{\footrulewidth}{0pt} % Remove footer underlines
\setlength{\headheight}{13.6pt} % Customize the height of the header

\numberwithin{equation}{section} % Number equations within sections (i.e. 1.1, 1.2, 2.1, 2.2 instead of 1, 2, 3, 4)
\numberwithin{figure}{section} % Number figures within sections (i.e. 1.1, 1.2, 2.1, 2.2 instead of 1, 2, 3, 4)
\numberwithin{table}{section} % Number tables within sections (i.e. 1.1, 1.2, 2.1, 2.2 instead of 1, 2, 3, 4)

\setlength{\baselineskip}{1.5em}
\linespread{1.0}
\setlength{\parskip}{1em}
\setlength\parindent{0pt} % Removes all indentation from paragraphs - comment this line for an assignment with lots of text

%----------------------------------------------------------------------------------------
%	TITLE SECTION
%----------------------------------------------------------------------------------------

\newcommand{\horrule}[1]{\rule{\linewidth}{#1}} % Create horizontal rule command with 1 argument of height

\title{	
\normalfont \normalsize 
\textsc{Iowa State University, Computer Science Department} \\ [25pt] % Your university, school and/or department name(s)
\horrule{0.5pt} \\[0.4cm] % Thin top horizontal rule
\huge Computer Science 511 \\ Assignment 1 \\ % The assignment title
\horrule{2pt} \\[0.5cm] % Thick bottom horizontal rule
}

\author{Kun Ji} % Your name

\date{\normalsize\today} % Today's date or a custom date

\begin{document}

\maketitle % Print the title

%----------------------------------------------------------------------------------------
\section{CLRS 23.1-7}
First, we show that any suitable subset of edges is a tree. Assume there exists some subset of edges which is not a tree, it must contain at least a cycle. We can remove some edges to obtain a spanning tree that has $n - 1$ vertices. Because all edges are of positive weights, the total weight of the new spanning tree would become smaller and this is a contradiction.\par
If negative edge weights are allowed, the conclusion would not hold. Consider a graph $G$ that has three vertices and three edges with weights 1, 1 and -2, the optimal subset of edges comprises all the three edges and hence does not form a tree.
%----------------------------------------------------------------------------------------

%----------------------------------------------------------------------------------------
\section{CLRS 23.1-9}

\paragraph{Proof} Suppose the minimum spanning tree of $G^{'}$ is $T^{"}$ so that $\omega(T^{"}) < \omega(T^{'})$. Let $S$ be the subset of edges $(T - T^{'})$. Then we can combine $T^{"}$ and $S$ to construct a new minimum spanning tree of G; the edges cross the cut ($V^{'}$,$V$-$V^{'}$) are still light edges, the other edges in $(T - T^{'})$ are not affected by the division and still a part of minimum spanning tree. The total weight of the new MST is 
$\omega(T^{"})+\omega(S) < \omega(T^{'})+\omega(S) = \omega(T)$. Thus, we get a contradiction because $T$ is a minimum spanning tree of $G$. Hence $T^{'}$ must be a minimum spanning tree of $G^{'}$.
%----------------------------------------------------------------------------------------

%----------------------------------------------------------------------------------------
\section{CLRS 23.1-10}

\paragraph{Proof} Since edge $(x,y) \in T$,  and removing $(x,y)$ breaks $T$ into two components $T^{'}$ and $T - \{(x,y)\} - T^{'}$. Let $S$ be the vertices in $T^{'}$ and its complement is $V - S$. After the weight of $(x,y)$ is decreased, $(x,y)$ is still a light edge crossing $(S, V - S)$ and thus must be in some MST of $G^{'}$. Besides, the two components of $T$ are still the MST's of the two sub-graphs which are separated by the edges cross $(S, V - S)$. Since $(x,y)$, $T^{'}$ and $T - \{(x,y)\} - T^{'}$ still are in the new MST, we show that $T$ remains a MST for $G$. 

%----------------------------------------------------------------------------------------

\section{KT 4-19}   

\paragraph{Algorithm description.} We can construct an algorithm analogous to Kruskal's algorithm except that at each step our algorithm adds a maximum-weight edge if possible. This greedy strategy manages a set of edges $A$, maintaining the following loop invariant:
\begin{center}
Prior to each iteration, $A$ is a subset of the tree we want. 
\end{center}
At each step, we determine an edge ($u, v$) that does not violate this invariant. Initially, $A$ is empty and we create |$V$| trees, one containing each vertex. And at each iteration, the edge to be added must be a maximum-weight edge which connects two components in the forest.
Thus, the solution can be obtained after |$V - 1$| steps.

\begin{algorithm}[H] 
\SetAlgoLined
\SetNoFillComment
\DontPrintSemicolon
\KwIn{Connected, undirected graph $G=(V,E)$ with edge weight function $w$.}
\KwOut{A spanning tree $A$ which achieves best achievable bottleneck rate for any path in $G$.}
$A = \emptyset$ \;
\ForEach{$v \in V$}{\texttt{MAKE-SET}($v$)}
sort the edges of $G.E$ into descending order by weight $w$ \;
\ForEach{edge $(u, v) \in G.E$, taken in descending order }{
	\If{\texttt{FIND-SET}($u$) $\neq$ \texttt{FIND-SET}($v$)}{
		$A = A \cup \{(u, v)\}$ \;
		\texttt{UNION}($u, v$) 
	}
}
\Return $A$ 
\caption{\texttt{Best-Bottlenect-Rate}$(G,w)$}
\end{algorithm}

\begin{lemma} Given two spanning trees $T_{1}$, $T_{2}$ and the maximum-weight edge that connects them is $(u, v)$, where $u \in T_{1}$ and  $v \in T_{2}$, and $\omega(u, v)$ is less than the weight of any edge in $T_{1}$ and $T_{2}$. Then for any path $P$ in $T$ that passes $(u, v)$, $b(P)$ = $\omega(u, v)$.
\end{lemma}

\begin{proof} The conclusion is intuitive.
\end{proof}

\begin{theorem} Assume a forest $F$ has components $C_{1}, C_{2}, \cdots, C_{k}$ and the spanning tree of any component in $F$ is a part of the optimal spanning tree $T$, the maximum-weight edge that can connect two components in the forest must belong to $T$. 
\end{theorem}

\begin{proof} It can be proved by induction on number of iterations.
Initially, there are $|V| - 1$ trees. Assume the maximum-weight edge is $(u, v)$, $b(u \to v)$ is apparently $\omega(u, v)$. Then we can combine $u$ and $v$ into a component because the selection of next edge does not affect the value of $b(u \to v)$.
 
Suppose the claim holds after $k - 1$ iterations. Then the vertices in the same component are connected with each other and the edges in the component belongs to $T$. Now assume the maximum-weight edge that can connect any two components is $(u, v)$ but the edge does not belong to $T$. Then for any pair of nodes $u$, $v$ which belong to two different components, there exists a path $P$ that does not contain $(u, v)$ and $b(P) < \omega(u, v)$. This conclusion contradicts with the assumption that $b(P) \geq \omega(u, v)$.\par Therefore, $(u, v)$ must belong to $T$. This claim holds when edge weights are not all distinct because we can add the edges with the same weight consecutively.
\end{proof}
\paragraph{Running time.} The time to sort the edges in line 4 is $O(E lgE)$. The \emph{for} loop of lines 5-8 performs $O(E)$ \texttt{FIND-SET} and \texttt{UNION} operations. Each operation takes $O(lgE)$ time. Observing that $|E| < |V|^{2}$, we have $lgE = O(lg V)$, and hence the running of time of this algorithm is $O(ElgV)$.

%----------------------------------------------------------------------------------------
\section{KT 4-22}
\paragraph{Answer.} It is possible $T$ is not a minimum spanning tree  and a counterexmaple is provided in Figure \ref{fig:d1}. To illustrate why this conclusion is incorrect, let's assume there are two minimum spanning trees in $G$: $T$ and $T^{'}$. And the edges in them that cross some cut ($S$, $V-S$) are $e$ and $e^{'}$, respectively. Thus, we can construct a spanning tree $T^{"}$ by add $e^{'}$ to $T$ and delete another edge that does not cross ($S$, $V-S$). It is easy to verify $T^{"}$ is a spanning tree that satisfies all the necessary conditions. However, the weight of deleted edge could be larger than the weight of $e^{'}$. It is possible because $e^{'}$ is only a light edge cross ($S$, $V-S$) but its weight could be significantly large. 
\begin{figure}[h]
\centering
\includegraphics[scale=1.0]{diag1.eps}
\caption{\textbf{(a)} A simple connected, undirected graph $G=(V,E)$. \textbf{(b) and (c)} Two MST's in $G$ can be obtained easily. \textbf{(d)} A spanning tree in which every edge belongs to one of the MST's. Apparently, this spanning tree is not a MST.}
\label{fig:d1}
\end{figure}



\end{document}